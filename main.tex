%%%%%%%%%%%%%%%%%%%%%%%%%%%%%%%%%%%%%%%%%
% "ModernCV" CV and Cover Letter
% LaTeX Template
% Version 1.11 (19/6/14)
%
% This template has been downloaded from:
% http://www.LaTeXTemplates.com
%
% Original author:
% Xavier Danaux (xdanaux@gmail.com)
%
% License:
% CC BY-NC-SA 3.0 (http://creativecommons.org/licenses/by-nc-sa/3.0/)
%
% Important note:
% This template requires the moderncv.cls and .sty files to be in the same 
% directory as this .tex file. These files provide the resume style and themes 
% used for structuring the document.
%
%%%%%%%%%%%%%%%%%%%%%%%%%%%%%%%%%%%%%%%%%
 
%----------------------------------------------------------------------------------------
%	PACKAGES AND OTHER DOCUMENT CONFIGURATIONS
%----------------------------------------------------------------------------------------


\documentclass[11pt,letterpaper,sans]{moderncv} % Font sizes: 10, 11, or 12; paper sizes: a4paper, letterpaper, a5paper, legalpaper, executivepaper or landscape; font families: sans or roman


\moderncvstyle{classic} % CV theme - options include: 'casual' (default), 'classic', 'oldstyle' and 'banking'
\moderncvcolor{blue} % CV color - options include: 'blue' (default), 'orange', 'green', 'red', 'purple', 'grey' and 'black'

\usepackage{lipsum} % Used for inserting dummy 'Lorem ipsum' text into the template
\usepackage{amsfonts}
\usepackage{amsmath}
\usepackage[scale=0.8]{geometry} % Reduce document margins
%\setlength{\hintscolumnwidth}{3cm} % Uncomment to change the width of the dates column
\setlength{\makecvtitlenamewidth}{10cm} % For the 'classic' style, uncomment to adjust the width of the space allocated to your name
%----------------------------------------------------------------------------------------
%	NAME AND CONTACT INFORMATION SECTION
%----------------------------------------------------------------------------------------

\firstname{Anselm} % Your first name
\familyname{Vossen} % Your last name

% All information in this block is optional, comment out any lines you don't need
%\title{Curriculum Vitae}
%\address{Address}{City, State Zip}
\mobile{(217) 819-6176}
\phone{(919)660-7593}
\email{anselm.vossen@duke.edu}
%\homepage{www.linkedin.com/in/anmol-sikka-696537103/} {LinkedIn} % The first argument is the url for the clickable link, the second argument is the url displayed in the template - this allows special characters to be displayed such as the tilde in this example
\homepage{https://phy.duke.edu/people/anselm-g-vossen}{}
%\extrainfo{DOB: September 15, 1997}
%\photo[70pt][0.4pt]{pictures/House} % The first bracket is the picture height, the second is the thickness of the frame around the picture (0pt for no frame)
%\quote{"A witty and playful quotation" - John Smith}
%----------------------------------------------------------------------------------------

\begin{document}

\makecvtitle % Print the CV title
%----------------------------------------------------------------------------------------
%	EDUCATION SECTION
%----------------------------------------------------------------------------------------

\section{\textbf{Education}}

\cventry{2005--2008}{Ph.~D.~in Physics}{\textit{Faculty for Math and Physics, \link[University of Freiburg]{https://www.physik.uni-freiburg.de/}}}{}{Dissertation "Transverse Spin Asymmetries at the COMPASS Experiment",  \textbf{summa cum laude}}{}

\cventry{2000--2004}{Master~in Computer Science}{\textit{Department of Computer Science, \link[University of Freiburg]{https://www.informatik.uni-freiburg.de/?set_language=en}}}{}{Dissertation "Invariance in Kernel Methods based on Integration over Transformations"}{}
\cventry{}{Degree with Distinction}{}{}{}{}

%\cventry{2001--2002}{Scholarship for Exchange Program in Computer Science}{Northern Arizona University}{}{Team leader in senior research project "Distributed Image Reformation" for Lockheed-Martin, award for excellent academic achievement}{}
\section{Academic Employment}
\subsection{Duke University/Jefferson Lab}
\cvitem{2018--present}{\textbf{Assistant Professor} joint appointment with Jefferson Lab}
%\cventry{January 2018--present}{Assistant Professor}{Duke University}{}{}{}
\subsection{Indiana University}
\cvitem{2017}{\textbf{Associate Scientist}}
\cvitem{2010--2017}{\textbf{Assistant Scientist}}
%\cventry{2018-2019}{Associate Research Scientist}{Indiana University}{}{}{}
%\cventry{November 2010--2018}{Assistant Research Scientist}{Indiana University}{}{}{}
\subsection{University of Illinois at Urbana-Champaign}
\cvitem{2008--2010}{\textbf{Postdoctoral Research Associate}}
%----------------------------------------------------------------------------------------
%	INTERESTS SECTION
%----------------------------------------------------------------------------------------


% \renewcommand{\listitemsymbol}{-~} % Changes the symbol used for lists
% \cvlistdoubleitem{Piano}{}
% \cvlistitem{Baseball}

%----------------------------------------------------------------------------------------
%	Achievements SECTION
%----------------------------------------------------------------------------------------

%\section{Scholastic and Curricular Achievements}


\section{Awards}

\cventry{2008}{Ph.D thesis "summa cum laude"}{}{}{}{}{}
\cventry{2005--2008}{Scholarship from the DFG Research Training Group "Physik an Hadron Beschleunigern" (Physics at Hadron Accelerators)}{}{}{}{}
\cventry{2005}{Award from the Society of German Engineers (VDI) for outstanding scientific achievement with the Diploma thesis}{}{}{}{}{}
\cventry{2004}{Master's degree in Computer Science with Distinction}{}{}{}{}{}
\cventry{2003--2008}{e-fellows scholarship, sponsored by McKinsey, Deutsche Telekom and others}{}{}{}{}
\cventry{2001--2002}{Scholarship for Exchange Program in Computer Science}{Northern Arizona University}{}{Team leader in senior research project "Distributed Image Reformation" for Lockheed-Martin, award for excellent academic achievement}{}


\section{Service and Leadership}
\cvitem{2020}{Internal PAC proposal reviewer CLAS12}
\cvitem{2020}{Member CLAS12 data preservation task-force}
\cvitem{2020--present}{\textbf{Co-convener for SIDIS processes for the EIC User groups Yellow Report process}}
\cvitem{2019--present}{\textbf{Co-convener Jefferson Lab EIC SIDIS task force}}
\cvitem{2019--present}{\textbf{Reviewer for Physical Review Letters}}
\cvitem{2014--2017}{Member of the writing committee for the pp/pA physics case at STAR }
\cvitem{2015}{Co-led effort to write white paper on $e^+e^-$ fragmentation as input for the current long range plan of the nuclear physics community in the US available at: \link[https://www.phy.anl.gov/nsac-lrp/Whitepapers/StudyOfFragmentationFunctionsInElectronPositronAnnihilation.pdf]{https://www.phy.anl.gov/nsac-lrp/Whitepapers/StudyOfFragmentationFunctionsInElectronPositronAnnihilation.pdf}}
\cvitem{2015}{\textbf{Proposal Reviewer for the Office of Nuclear Physics in the Department of Energy Office of Science}}
\cvitem{2014--2015}{Member of Panel for STAR coding standards and incorporation of C++ 11}
\cvitem{2013 -- 2015}{Member STAR trigger board tasked with implementing the physics goals for RHIC runs}
\cvitem{2013}{Member of the Scientific Program Committee, STAR collaboration meeting at LBNL}
\cvitem{2013--2016}{\textbf{Co-Convener of the Spin Physics Working Group of the STAR Experiment}}
\cvitem{2011--2015}{Member of the eSTAR task force}
\cvitem{2011--2017}{\textbf{Project manager Belle II BKLM Front-End Electronics}}
\cvitem{2011--2014}{\textbf{Software Coordinator for the Forward GEM Tracker project at STAR}}
\cvitem{2011 \& 2015}{Period Coordinator for the STAR experiment at RHIC}
\cvitem{2010--2012}{Member of the RHIC Users Executive Committee}
\cvitem{2008}{Project management for the PHENIX forward upgrade: Created infrastructure for tracking and progress monitoring}
\cvitem{2005--2008}{Speaker for the students in the DFG Research Training Group "Physik an Hadron Beschleunigern" (Physics at Hadron Accelerators)}


\cvitem{}{Member of the Committee for Graduate Admissions and Financial Support at Indiana University}
\cvitem{2011}{Poster Judge at annual RHIC/AGS Users' meeting 2011}
\cvitem{2011--present}{Institutional board member Belle and Belle II Experiments}
\cvitem{2016--2017}{Institutional board member EIC User's group}
\cvitem{2010--present}{Member of several paper review and drafting committees at RHIC experiments and at Belle}


\section{Funding}
\cvitem{}{(only major awards, excludes external support for workshops and conferences etc)}
\cvitem{2019--2022}{Flavor Physics with the Belle II Detector at Duke University (DOE HEP), \$305,000, \textbf{Sole PI} }
\cvitem{2018--2023}{Novel Experimental Probes of QCD in SIDIS and e+e- Annihilation (DOE NP Early Career), \$750,000, \textbf{Sole PI} }
\cvitem{2016--2019}{Studies in Nuclear Physics and Fundamental Interactions at Indiana University (NSF), \$5,400,000 with 8 Co-PIs total }
\cvitem{2015--2018}{Funding for Postdoctoral Researcher for Belle II MuID commissioning (DOE Belle II Project), \$160,000, \textbf{Sole PI} }
\cvitem{2013--2016}{Studies in Nuclear Physics and Fundamental Interactions at Indiana University (NSF), \$7,187,853 with 9 Co-PIs total }
\cvitem{2012--2016}{Design, development, fabrication and testing of the circuit boards for the Belle II Resisitive Plate Chamber (RPC) readout system, development of an active High Voltage Divider Board and signal amplification circuit for use with MCP-PMT of the iTOP detector for Belle II, mechanical design of the front-end board stack of the iTOP detector. (DOE Belle II Project), \$1,300,000, \textbf{Sole PI}}
\cvitem{2011--2013}{Studies in Nuclear Physics and Fundamental Interactions at Indiana University (NSF), \$4,175,000 with 10 Co-PIs total.}

%----------------------------------------------------------------------------------------
%	Techincal Projects SECTION
%----------------------------------------------------------------------------------------


%----------------------------------------------------------------------------------------
%	AWARDS SECTION
%----------------------------------------------------------------------------------------
%\section{Certifications}

%\cventry{April 2015 -- April 2017}{Certified LabVIEW Associate Developer}{\textsc{National Instruments}}{}{}{}


%----------------------------------------------------------------------------------------
%	Research SECTION
%----------------------------------------------------------------------------------------
\newpage
\section{Research}

\cvitem{}{My research is focused on experimental particle and nuclear physics at facilities in the US and Japan. It is organized along three interconnected themes listed in the following. I made significant contributions to each of them.
In particular, I am an expert on the study of hadronization of quarks into hadrons and on the spin structure of protons. 
I have written a well received review on hadronization and are currently working on an invited review on the transverse spin structure.
}
\subsection{Study of the strong force in hard scattering off nuclei and in electron-positron annihilation}
\cvitem{}{Quantum-Chromodynamics is the theory describing the strong interaction from which most of the properties of the visible mass emerge. It can be studied by probing hadrons in hard scattering experiments or in electron-positron annihilation where hadrons are produced from the QCD vacuum.
}


\cvitem{2018--present}{CLAS12 at JLab
\begin{itemize}
    \item \textbf{First extraction of di-hadron beam spin asymmetries sensitive to quark-gluon correlations from CLAS12 data.}
    \item Studies of $\Lambda$ production in the SIDIS current fragmentation region. Comparison with $e^+e^-$ will allow an unique test of universality of QCD.
\end{itemize}
}
\cvitem{2011--present}{Belle II experiment at KEK
\begin{itemize}
    \item Responsible for the development of readout electronics for the Barrel KLong and muon ID system (BKLM) and Belle II PID (2011--2017). Contributions to commissioning of the BKLM as well as BKLM software development
\end{itemize}}
\cvitem{2010--2017}{STAR Experiment at RHIC:
\begin{itemize}
    \item Co-Convener of the spin physics working group of the STAR collaboration: Responsible for the scientific output of the spin physics program.
    \item Software Coordinator for the newly installed Forward GEM Tracker (FGT) at STAR: Responsible for the development of all online and offline software 2011-2014.
    \item Measurement of transverse spin structure of the nucleon with the STAR detector at $\sqrt{s}=200$~GeV and $500$~GeV: \textbf{ First measurement of spin dependent two hadron correlations in transversely polarized $pp$ collisions leading to first global extraction of transversity}. First direct measurement of the transversity distributions of quarks in proton-proton scattering. %Publication in Phys.Rev.Lett. and Phys.Lett. 
\end{itemize}
}
\cvitem{2008--present}{Belle experiment at KEK
%Measurement of polarization dependent and integrated fragmentation functions
\begin{itemize}
    \item First measurement of the transverse polarization dependent Collins fragmentation function for neutral mesons  
    \item \textbf{First observation of transverse polarization of $\Lambda$ hyperons produced in $e^+e^-$}.
      \item First measurement of helicity dependent di-hadron correlations sensitive to local strong CP violation effects
    \item \textbf{First extraction of the di-hadron Interference Fragmentation Function needed to extract transversity from SIDIS and $pp$ measurements}
    \item Precision measurement of cross-sections of identified single hadrons and back-to-back pairs.
  %  \Precision measurements of fragmentation functions and local parity violating effects with the Belle detector at KEK, Japan 
\end{itemize}
}
\cvitem{2008--2010}{PHENIX Experiment at RHIC:
\begin{itemize}
    \item Development and commissioning of resistive plate chamber detectors for the fast muon trigger of the PHENIX experiment for the measurement of sea-quark polarization in the nucleon using W-bosons. 
    \item Responsible for the design and development of the upstream detector and for the integration of prototypes in the experiment and background measurements.
    \item Supervision of graduate and undergraduate students working on detector assembly and testing.
    \item	Measurement of transverse spin structure of the nucleon with the PHENIX detector using two hadron correlations.
    \item Introduction of unbinned maximum likelihood estimator to extract transverse single spin asymmetries.
    \item Principal author on PHENIX paper on forward transverse single spin asymmetries.
\end{itemize}
}
\cvitem{2004--2008}{COMPASS Experiment at CERN
\begin{itemize}
    \item Measurement of Collins and Sivers asymmetries and spin dependent two hadron correlations in $\mu$-deuteron SIDIS.
    \item Development of the statistical methods used to extract observables for all related COMPASS measurements.
\end{itemize}}



%\cite{Metz:2016swz}
\subsection{Tests of the standard model at the intensity frontier}

\cvitem{2008--present}{Belle
\begin{itemize}
    \item  Measurement of branching ratio of $B\rightarrow D^{(*)}\pi l \nu$.
\end{itemize}
}
\cvitem{2011--present}{Belle II
\begin{itemize}
    \item Measurement of semi-leptonic B decays.
    \item Search for new physics in the measurement of radiative penguin diagrams.
\end{itemize}
}
\subsection{Test of the QCD vacuum structure}
\cvitem{2019--present}{
\begin{itemize}
    \item Search for locally P-odd effects sensitive to interactions of instantons and sphalerons mediating fluctutations between QCD vacua with different winding numbers.
\end{itemize}
}
\cvitem{}{}



%\subsection{Belle \& Belle II detector}
%\cvitem{}{Measurement of spin dependent fragmentation functions and locally parity violating effects}
%\cvitem{}{First extraction and presentation of the di-hadron Interference Fragmentation Function. First author on the publication in Phys. Rev. Letters}
%\cvitem{}{Development and commissioning of resistive plate chamber detectors for the fast muon trigger for the PHENIX experiment for the measurement of sea-quark polarization in the nucleon using W-bosons. Responsible for the design and development of upstream detector and for integration of prototypes in the experiment and background measurements, supervision of graduate and undergraduate students working on detector assembly and testing.}
%\cvitem{}{Measurement of transverse spin structure of the nucleon with the PHENIX detector: Work with graduate students on the extraction of transversity from two hadron correlations at PHENIX.}
%\cvitem{}{introduction of unbinned maximum likelihood estimator to extract transverse spin asymmetry AN.}
%\cvitem{}{Cluster decomposition analysis for cluster AN extraction from the Muon Piston Calorimeter, principal author on PHENIX AN paper.}
%\cvitem{}{Work on planned forward upgrade of the PHENIX detector: Simulation of kinematics and amplitudes of transverse single spin asymmetries.}   
%\cvitem{}{Precision measurements of fragmentation functions and local parity violating effects with the Belle detector at KEK, Japan: Under A. Vossen’s leadership, Indiana University joined Belle and Belle II collaborations, develops front-end electronics for barrel muon detectors and imaging time of propagation detector as part of US Belle II project.}
%\cvitem{}{the analysis of Collins fragmentation function for 0 and  from Belle data.}
%\subsection{STAR detector}
%\subsection{PHENIX detector}
%\cvitem{}{}
%\subsection{COMPASS detector, CERN}
%\cvitem{}{Measurement of Collins and Sivers asymmetries and spin dependent two hadron correlations}
%\subsection{Computer Science}
%\cvitem{}{Pattern recognition in Images using multivariate methods in particular Support Vector Machines}
%\cvitem{}{Invariance in Kernel Functions by Integrations over Transformations}
%\cvitem{}{Computing Optical Slices with Structured Illumination}

\section{Conferences and Workshops Organized}
\cvitem{April 2019}{Co-organizer of the session "Quark-Gluon Correlations" at the 8th Workshop of the APS Topical Group on Hadronic Physics, Denver, CO}
\cvitem{March 2019}{\textbf{Chair of the Organizing Committee for the "Workshop on Novel Probes of the Nucleon Structure in SIDIS, e+e- and pp" (FF2019), Durham, NC}}
\cvitem{October 2018}{Co-Organizer of the Workshops "Hadron Structure I+II", 5th Joint Meeting of the APS Division of Nuclear Physics and the Physical Society of Japan, Waikoloa, HI}
\cvitem{September 2016}{\textbf{Co-Chair of the Organizing Committee, "22nd International Spin Symposium" (SPIN 2016), Urbana, IL}}
\cvitem{December 2013}{\textbf{Co-Chair of the Organizing Committee for the "Indiana-Illinois Workshop on Fragmentation Functions", Bloomington, IN}}
\cvitem{October 2013}{Member of the physics program committee for the STAR collaboration meeting, October 2013, Berkeley, CA}
\cvitem{July 2012}{Co-Organizer for the workshop "Forward Physics at RHIC", July 2012, BNL}
\cvitem{December 2011}{Member of the program committee for the STAR upgrade workshop, December 2011, Los Angeles, CA}
\cvitem{May 2011}{Co-Organizer of the Workshop "Opportunities for Drell-Yan Physics at RHIC", May 2011, BNL}
\cvitem{June 2010}{Co-Organizer of the Spin Workshop during the 2010 RHIC/AGS Users' Meeting}


\section{Mentoring}
\subsection{Postdocs}
\cvitem{2019--present}{Frank Meier -- Flavor physics at Belle II/Belle II BKLM commissioning}
\cvitem{2019--present}{Christopher Dilks -- Di-hadron studies at CLAS12}
\cvitem{2015--2018}{Yinghui Guan -- Transverse polarization of $\Lambda$ at Belle/Belle II BKLM commissioning}
\cvitem{2012--2014}{Mike Skoby -- Di-hadron studies at STAR}
\subsection{Graduate Students}
\cvitem{2019--present}{Simon Schneider -- "Transverse polarization of $\Lambda$ hyperons in CLAS12"}
\cvitem{2019--present}{Katherine Parham -- "Identified di-hadron fragmentation at Belle II"}
\cvitem{2011--2016}{Hairong Li -- "Azimuthal asymmetries of pairs of back-to-back  charged and neutral pions, as well as $\eta$ mesons in $e^+e^-$ annihilation"}
\cvitem{2013}{Member of the thesis committee of Brian Page, thesis title: "Di-jet Cross-Section and Double Spin Asymmetry at Mid-Rapidity from Polarized p+p Collisions at sqrt{s}=200 GeV at RHIC"}
\subsection{Undergraduate Students}
\cvitem{Summer 2019}{Matthew McEneaney (REU student at Duke) -- Efficiency of $\Lambda$ reconstruction at CLAS12}

\cvitem{2008--2010}{Supervision of undergraduate and graduate students at BNL
\begin{itemize}
    \item Supervision of graduate students building resistive plate chambers
    \item Mentoring of graduate students in PHENIX on analysis topics
    \item Supervision of undergraduate students from Muhlenberg College, Abilene Christian University and University of Illinois, working on the forward upgrade, some results were presented at the APS April Meeting, 2009
\end{itemize}}
\cvitem{2000--2008}{Teaching Assistant, Department of Physics and Computer Science, University of Freiburg
\begin{itemize}
    \item Teaching and grading of several intermediate and advanced physics labs
    \item Teaching Assistant for lectures and labs in the area of software engineering and pattern recognition.
\end{itemize}}
\newpage
\section{Teaching}
\cvitem{Spring 2019}{Instructor, Physics 806: Radiation Detection, Duke University
\begin{itemize}
    \item Developed syllabus, lectures and exam
\end{itemize}
}
\cvitem{Spring 2018}{Discussion leader, Physics 151: Introductory Mechanics}
\cvitem{Fall 2011}{Instructor, P309: Intermediate Laboratory, Indiana University
\begin{itemize}
    \item Developed syllabus, lecture, supervised experiments, graded students work
\end{itemize}
}



\section{Seminar, Lectures and Talks}
\cvitem{November 2019}{\textbf{Seminar} "Novel experimental probes of QCD in $pp$, SIDIS and $e^+e^-$ annihilation", University of North Carolina, Wilmington }
\cvitem{October 2019}{Talk, "Results on light quark fragmentation from Belle", 2019 Fall Meeting of the APS Division of Nuclear Physics, Crystal City, VA}
\cvitem{September 2019}{\textbf{Invited Talk} "Fragmentation at the EIC", 9th International Conference on Physics Opportunities at an ElecTron-Ion-Collider (POETIC 2019), Berkeley, CA}
\cvitem{August 2019}{\textbf{Invited Talk} "Fragmentation Functions", The 11th Workshop on Hadron Physics in China and Opportunities Worldwide, Tianjin, China}
\cvitem{April 2019}{\textbf{Seminar} "Novel experimental probes of QCD in SIDIS and $e^+e^-$ annihilation", University of Kentucky}
\cvitem{April 2019}{Talk "Experimental results on quark-gluon correlations", Workshop of the APS Topical Group on Hadronic Physics, Denver CO}

\cvitem{October 2018}{Talk "Di-hadron correlations in SIDIS at CLAS12", 5th Joint Meeting of the APS Division of Nuclear Physics and the Physical Society of Japan, Waikoloa, HI }

\cvitem{October 2018}{\textbf{Invited Talk} "Di-hadron and polarized lambdas as novel probes of the nucleon structure", INT 2018}
\cvitem{September 2018}{\textbf{Invited Talk} "Recent experimental results and outlook from e+e- annihilation", Quantum Entanglement at Collider Energies, Stony Brook, NY}
\cvitem{June 2018}{\textbf{Invited Lectures} "Hadron Structure Physics", National Nuclear Physics Summer School, Yale, CT}
\cvitem{May 2018}{\textbf{Invited Talk} Light Quark Fragmentation Studies at the B-Factories, CIPANP 2018 - Thirteenth Conference on the Intersections of Particle and Nuclear Physics, Palm Springs, CA}
\cvitem{May 2018}{Talk "First Collisions at Belle II", CIPANP 2018 - Thirteenth Conference on the Intersections of Particle and Nuclear Physics, Palm Springs, CA }
\cvitem{May 2018}{Talk "Novel Experimental Probes of QCD in SIDIS and $e^+e^-$ Annihilation", Palm Springs, CA  }
\cvitem{May 2018}{\textbf{Invited Talk} "Recent results and prospects for hadronic physics at Belle and Belle II", Lightcone 2018, JLab}
\cvitem{December 2017}{\textbf{Invited Talk} "Recent Results on Fragmentation Functions from e+e- facilities", Transversity 2017, Frascati, Italy}
\cvitem{September 2017}{Talk "Prospects for hadronic physics at Belle II", Hadronic Physics with Lepton and Hadron Beams, Jefferson Lab}
\cvitem{April 2017}{Talk "Observation of Transverse Lambda Polarization in e+e- Annihilation at Belle" DIS 2017 Birmingham, UK}
\cvitem{November 2017}{\textbf{ Seminar} "Hadronization Studies at Belle", CERN}
\cvitem{November 2016}{\textbf{ Invited Talk} "Parton Fragmentation Functions", Parton Radiation and Fragmentation from LHC to FCC-ee, CERN}
\cvitem{November 2016}{\textbf{ Invited Talk} "Fragmentation Functions from Belle", Parton Radiation and Fragmentation from LHC to FCC-ee, CERN}
\cvitem{October 2016}{\textbf{Seminar} "Hadronization Studies at Belle", University of Michigan}
\cvitem{March 2016}{\textbf{ Seminar} "Exploring QCD Dynamics and Proton Structure in Polarized p-p Scattering and $e^+e^-$ Annihilation", William \& Mary}
\cvitem{February 2016}{\textbf{Seminar} "Exploring QCD Dynamics and Proton Structure in Polarized p-p Scattering and e+e- Annihilation", Virginia Tech}
\cvitem{October 2015}{\textbf{Invited Talk} "Fragmentation Functions in e+e- Experiments", ECT* workshop "From 1D fragmentation to 3D correlated fragmentation functions, Trento, Italy}
\cvitem{September 2015}{\textbf{Invited Talk} "Asymmetry measurements in $e^+e^-$: methods, open points and perspectives",  TMDe2015, a path towards TMD extraction, Trieste, Italy}
\cvitem{May 2015}{\textbf{Invited plenary talk} "Present and future directions in disentangling the spin of the nucleon", Twelfth Conference on the Intersections of Particle and Nuclear Physics (CIPANP2015), Vail, CO}
\cvitem{May 2015}{\textbf{Seminar} "Partonic Structure beyond Helicity Distributions: Recent results from the Transverse Spin Program at STAR and Fragmentation Function Measurements at Belle and Belle II", University of Illinois, Urbana-Champaign}
\cvitem{April 2015}{Talk "Measurement of Fragmentation Functions in e+e- Annihilation at Belle", XXIII. International Workshop on Deep-Inelastic Scattering and Related Subjects (DIS2015), Dallas, TX}
\cvitem{April 2015}{\textbf{Invited Talk} "Measurement of Fragmentation Functions in e+e- Annihilation at Belle \& Belle II", APS April Meeting, Baltimore, MD}
\cvitem{April 2015}{\textbf{Invited Talk} "Recent progress in transverse spin physics", Workshop of the APS Topical Group on Hadronic Physics, Baltimore, MD}
\cvitem{June 2014}{\textbf{Invited Talk} "Transverse Single Spin and Azimuthal Asymmetries in Hadronic Collisions
at STAR", 4th International Workshop on Transverse Polarisation Phenomena in Hard Processes, Chia, Italy}
\cvitem{March 2014}{\textbf{Invited Talk} "Recent Results on the Measurement of Fragmentation Functions in $e^+e^-$ annihilation", QCD Evolution Workshop 2014, Santa Fe, NM}
\cvitem{March 2014}{\textbf{Seminar} "Novel Probes of the Transverse Nucleon Spin Structure in Polarized Proton-Proton Collisions", Jefferson Lab}
\cvitem{June 2013}{\textbf{Invited Talk} "Fragmentation Studies in e+e- and p+p", Workshop Structure of Nucleons and Nuclei, Como, Italy} 
\cvitem{April, 2013}{\textbf{Invited plenary talk} "Experimental Measurements of Transverse Momentum Dependent Nucleon Distribution Functions", Workshop of the APS Topical Group on Hadronic Physics, Denver, CO}
\cvitem{March 2013}{\textbf{Seminar} "Measuring Transverse Quark Polarization with Spin-Dependent Fragmentation Functions", Brookhaven National Laboratory, Upton, NY}
\cvitem{February 2013}{\textbf{Seminar} "Novel Probes of the Transverse Spin Structure", University of Connecticut, Storrs, CT}
\cvitem{February 2013}{\textbf{Seminar} "Novel Probes of the Transverse Spin Structure", Temple University, Philadelphia, PA}
\cvitem{November 2012}{\textbf{Seminar} "Recent Results in Spin Physics at STAR and Belle", University of Freiburg, Germany}
\cvitem{November 2012}{\textbf{Invited Talk} "Measuring Transversity at COMPASS and RHIC with Spin Dependent Fragmentation Functions", Workshop on Fragmentation Functions and QCD, RIKEN, Wako, Japan}
\cvitem{November 2012}{\textbf{Invited Talk} "Measurement of Polarized Fragmentation Functions in $e^+-e^-$ at Belle", Workshop on Fragmentation Functions and QCD, RIKEN, Wako, Japan}
\cvitem{June 2012}{\textbf{Invited Talk} "Transversity from Di-hadron correlations", RHIC \& AGS Annual User's meeting 2012, BNL}
\cvitem{June 2012}{Talk "Measuring transversity in polarized $pp$ collisions at $\sqrt{s}=200$~GeV with the STAR detector", CIPANP 2012, St. Petersburg, Fl}
\cvitem{May 2012}{\textbf{Invited Talk} "Recent Experimental Results from RHIC Spin and the Measurement of Fragmentation Functions at Belle", QCD Evolution Workshop 2012, Jefferson Lab}
\cvitem{April 2012}{\textbf{Invited Talk} "Prospects for detecting P-odd effects at Belle", P- and CP-odd Effects in Hot and Dense Matter, RIKEN BNL Research Center Workshop, BNL}
\cvitem{September 2011}{\textbf{Invited Talk}  "Measurement of Interference Fragmentation Functions in e+e- and Di-Hadron Correlations at PHENIX and STAR", Miniworkshop on Dihadron Fragmentation Functions, Pavia, Italy}
\cvitem{August 2011}{\textbf{Invited Talk} "Measurement of (Interference) Fragmentation Functions in e+e- and Di-Hadron Correlations at SIDIS and p+p", Transversity 2011, Veli Losinj, Croatia}
\cvitem{June 2011}{\textbf{Invited Talk} "Measurement of Interference Fragmentation Functions at Belle and Extraction of Transversity from Di-Hadron Correlations at SIDIS and p+p", RHIC \& AGS Annual Users' Meeting, 2011, BNL}
\cvitem{March 2011}{\textbf{ Seminar} "Spin Dependent Fragmentation Functions at Belle and Belle II: Tools to analyze nucleon structure and the QCD vacuum", SUNY Stony Brook, NY}
\cvitem{April 2011}{Talk "First Measurement of Interference Fragmentation Function at Belle", DIS 2011, Newport News,  VA}
\cvitem{February 2011}{\textbf{Invited Talk} "Transverse Spin Asymmetries at Phenix", 27th WinterWorkshop on Nuclear Dynamics,  Winter Park, CO}
\cvitem{June 2010}{\textbf{Lecture} for summer students "Spin", BNL}
\cvitem{May 2010}{\textbf{Invited Talk} "Future of Transverse Spin at Phenix", RHIC Spin Collaboration Meeting, Iowa State University, Ames, IA}
\cvitem{April 2010}{\textbf{Invited Talk} "Results on charged Pion Correlations and prospects for detecting P-odd effects at Belle", P- and CP-odd Effects in Hot and Dense Matter, RIKEN BNL Research Center Workshop}
\cvitem{April 2010}{Talk "Measurement of Fragmentation Functions at Belle", DIS 2010, Florence, Italy}
\cvitem{April 2010}{Talk "Transverse Spin Results from Phenix", DIS 2010, Florence, Italy}
\cvitem{April 2010}{Talk "Future Phenix Calorimetry and Trigger Upgrades", DIS 2010, Florence, Italy}
\cvitem{April 2010}{\textbf{Seminar} "Measurement of Fragmentation Functions at Belle", University of Pavia, Italy}
\cvitem{February 2010}{\textbf{Seminar} "Novel Probes of Nucleon Spin Structure at Phenix", Indiana University, Bloomington, IN}
\cvitem{April 2010}{Talk "Measurement of the Proton Quark Structure from Parity Violating Forward Lepton Asymmetries in W Production in Phenix", APS April Meeting 2010, Washington, USA}
\cvitem{October 2009}{\textbf{ Seminar} "Measurement of the Interference Fragmentation Function at Belle", University of Freiburg, Germany, October 2009}
\cvitem{September 2009}{\textbf{Invited Talk} "Transverse Spin Dependent Di-Hadron Fragmentation Functions at Belle", DSPIN 2009 Conference,, Dubna, Russia}
\cvitem{June 2009}{\textbf{Invited Talk} "Measuring the interference and Collins fragmentation functions at BELLE", RHIC \& AGS Annual Users' Meeting, BNL}
\cvitem{October 2008}{Talk "Fragmentation Function Measurement at Belle", SPIN 2008, Charlottesville, Virginia, USA}
\cvitem{September 2007}{\textbf{Invited Talk} "Transverse spin-dependent azimuthal asymmetries in SIDIS", EINN 2007 Conference, Milos, Greece}
\cvitem{March 2007}{\textbf{Invited Talk} "Measurements of Transverse Spin Effects by the COMPASS Experiment", XLIInd Rencontres de Moriond, La Thuile, Italy}
\cvitem{March 2006}{\textbf{Invited Talks} "Feature Selection and Statistical Learning Basics", "Support Vector Machines", iCSC (inverted CERN School of computing), Theme Co-Coordinator, CERN}



\newpage
\section{Selected Publications (Primary Author)}
\cvitem{}{Citation count from \link[\color{blue}INSPIRE-HEP (link for complete list)]{http://inspirehep.net/search?ln=en&p=find+a+a+vossen&of=hb&action_search=Search&sf=earliestdate&so=d}, 8/15/2019}
\underline{The (in my opinion) five most significant publications are marked with a '$\rhd$'}\\

%\cvitem{}{
  \begin{itemize}

\item M. Anselmino, A. Mukherjee and A. Vossen, "Transversity and novel spin phenomena in hard inclusive collisions", inivted review to appear in Prog.\ Part.\ Nucl.\ Phys.
\item H.Li, A. Vossen \textit{   et al.} (Belle Collaboration), "Azimuthal asymmetries of back-to-back $\pi^\pm-(\pi^0,\eta,\pi^\pm)$ pairs in $e^+e^-$ annihilation", Phys.\ Rev.\ D \textbf{100} (2019) no.9, 092008
    \item $\rhd$ Y. Guan, A. Vossen \textit{   et al.} (Belle Collaboration), "Observation of Transverse $\Lambda/\bar{\Lambda}$ Hyperon Polarization in $e^+e^-$
 Annihilation at Belle", Phys.\ Rev.\ Lett.\  \textbf{ 122}, no. 4, 042001 (2019), \textbf{8 citations}
 \item A. Vossen \textit{   et al.} (Belle Collaboration), ``Measurement of the branching fraction of $B \rightarrow D^{(*)}\pi \ell\nu$ at Belle using hadronic tagging in fully reconstructed events,'', Phys.\ Rev.\ D \textbf{ 98}, no. 1, 012005 (2018), \textbf{4 citations}
  \item L. Adamczyk \textit{   et al.} (STAR Collaboration), "Transverse spin-dependent azimuthal correlations of charged pion pairs measured in p$^\uparrow$+p collisions at $\sqrt{s}$ = 500 GeV", Phys.\ Lett.\ B \textbf{ 780}, 332 (2018), \textbf{6 citations} 
    \item $\rhd$ A. Metz and A. Vossen, "Parton Fragmentation Functions", Prog.\ Part.\ Nucl.\ Phys.\  \textbf{ 91}, 136 (2016), \textbf{51 citations}
    \item $\rhd$ L. Adamczyk \textit{   et al.} (STAR Collaboration), "Observation of Transverse Spin-Dependent Azimuthal Correlations of Charged Pion Pairs in $p^\uparrow+p$ at $\sqrt{s}=200$ GeV,'', Phys.\ Rev.\ Lett.\  \textbf{ 115}, 242501 (2015), \textbf{25 citations}
    %\item di-hadrons at Compass Transverse spin effects in hadron-pair production from semi-inclusive deep inelastic scattering
\item A. Abdesselam et al. (Belle Collaboration) "Measurement of Azimuthal Modulations in the Cross-Section of Di-Pion Pairs in Di-Jet Production from Electron-Positron Annihilation", BELLE-CONF-1502, e-Print: arXiv:1505.08020 [hep-ex], \textbf{10 citations}
%     Azimuthal asymmetries of charged hadrons produced by high-energy muons scattered off longitudinally polarised deuteronsEur.Phys.J. C70 (2010) 39-49, 36 citations
%    Measurement of the Collins and Sivers asymmetries on transversely polarised protons
 %   Phys.Lett. B692 (2010) 240-246 (217 citation)
 %   Collins and Sivers asymmetries for pions and kaons in muon-deuteron DIS, Phys.Lett. B673 (2009) 127-135 303 citations\item E.C. Aschenauer et al., "The RHIC Spin Program: Achievements and Future Opportunities" (RHIC Spin White Paper 2012), e-Print: arXiv:1304.0079 [nucl-ex], 68 citations
\item E. Aschenauer et al., "The RHIC SPIN Program: Achievements and Future Opportunities" (RHIC Spin White Paper 2015), e-Print: arXiv:1501.01220, \textbf{72 citations}
\item A. Adare \textit{ et al.}  (PHENIX Collaboration), "Measurement of transverse-single-spin asymmetries for midrapidity and forward-rapidity production of hadrons in polarized p+p collisions at $\sqrt{s}=$200 and 62.4 GeV,'', Phys.\ Rev.\ D \textbf{ 90}, no. 1, 012006 (2014), \textbf{73 citations}
    \item M. Leitgab \textit{  et al.} (Belle Collaboration), ``Precision Measurement of Charged Pion and Kaon Differential Cross Sections in e+e- Annihilation at s=10.52  GeV,'' Phys.\ Rev.\ Lett.\  \textbf{ 111}, 062002 (2013), \textbf{61 citations}
    \item $\rhd$ C. Adolph  \textit{  et al.} (COMPASS Collaboration)  ``Transverse spin effects in hadron-pair production from semi-inclusive deep inelastic scattering,'', Phys.\ Lett.\ B \textbf{ 713}, 10 (2012), \textbf{81 citations}
    \item $\rhd$ A. Vossen \textit{   et al.} (Belle Collaboration), ``Observation of transverse polarization asymmetries of charged pion pairs in $e^+e^-$ annihilation near $\sqrt{s}=10.58$ GeV,'', Phys.\ Rev.\ Lett.\  \textbf{ 107}, 072004 (2011), \textbf{107 citations}
    \item M. Alekseev \textit{  et al.} (COMPASS Collaboration), ``Collins and Sivers asymmetries for pions and kaons in muon-deuteron DIS,', Phys.\ Lett.\ B \textbf{ 673}, 127 (2009), \textbf{303 citations}
\item A. Vossen, "Basics of Feature Selection and Statistical Learning for High Energy Physics" and "Support Vector Machines in High Energy Physics", Lectures from iCSC, CERN Yellow Reports, CERN-2008-002, (arXiv:0803.2344v1 [physics.data-an], arXiv:0803.2345v1 [physics.data-an]), \textbf{3 citations}
\item B. Haasdonk, A. Vossen, and H. Burkhardt, "Invariance in Kernel Methods by Haar-Integration Kernels", SCIA 2005, Scandinavian Conference on Image Analysis, pp. 841-851, Springer, 2005, \textbf{24 citations}
\end{itemize}
%}


%\cvitem{2011-2017}{Member of the graduate faculty at Indiana University
%\begin{itemize}
%    \item Advisor of graduate student Hairong Li working on the measurement of fragmentation functions at Belle.
%    \item Member of the thesis committee of Brian Page, thesis title: "Di-jet Cross-Section and Double Spin Asymmetry at Mid-Rapidity from Polarized p+p Collisions at sqrt{s}=200 GeV at RHIC"
%\end{itemize}




%----------------------------------------------------------------------------------------
%	Course Projects SECTION
%----------------------------------------------------------------------------------------

%\section{Course Projects}
%\cvitem{Autumn 2018}\textbf{{IGDTUW Database Management Project modelling the University : DBMS)}}
%\cvitem{}{\textit{Prof. DK Tayal, Department of Computer Science}}

%\cvitem{March 2017}\textbf{{Toy Store: Computer Science}}
%\cvitem{}{\textit{Ms.Meenu Kumar, Computer Science}}

%\cvitem{August 2017}\textbf{{Internet Banking App:Prototype}}
%\cvitem{}{\textit{Developed for Esya,IIITD}}

%----------------------------------------------------------------------------------------
%	Institute Positions SECTION
%----------------------------------------------------------------------------------------


%\cventry{June 2008--October 2010}{Postdoctoral Research Associate}{\textit{University of Illinois at Urbana-Champaign, Department of Physics}}{}{}{}

%----------------------------------------------------------------------------------------
%	Skills SECTION
%----------------------------------------------------------------------------------------

%\section{Technical Skills}
%
%\cvitem{Programming}{\textsc{Python, Java, C++, C, SQL, HTML}}
%\cvitem{Software}{\textsc{\LATEX, Protopie, Atom, MS Word, MS Excel, MS Powerpoint, Javasciprt}}
%\cvitem{Web Development}{\textsc{FrontEnd- HTML, CSS, Bootstrap, Javascript, jquery,BackEnd- Command Line, Python, Django}}
%\cvitem{Editing}{\textsc{Adobe Premiere Pro, Adobe Photoshop,Filmora Adobe Lightroom}}

%----------------------------------------------------------------------------------------
%	Coursework SECTION
%----------------------------------------------------------------------------------------

%\section{Coursework}

%\renewcommand{\listitemsymbol}{-~} % Changes the symbol used for lists

%\cvitem{Core Courses}{Discrete Mathematics, Database Management Systems, Data Structure, Object Oriented Programming, Analog and Digital Electronics, Advance Engineering Mathematics*, Computer Organization and Architecture*, Analysis and of Algorithms Operating System*,Object Oriented Software Engineering*}


%\cvitem{Lab Courses}{Database Management Systems Lab, Data Structure Lab, Object Oriented Programming using C++ and JAVA Lab, Analog and Digital Electronics Lab, Computer Organization and Architecture Lab*, Analysis and Design of Algorithms Lab Operating System Lab (using LINUX as Case Study)*,Object Oriented Software Engineering Lab*}

%\cvitem{}{}
%\cvitem{}{* -To be completed by Apr 2019}
%\cvlistdoubleitem{Basics of Electricity & Magnetism}{Computer Programming and Utilization}

%----------------------------------------------------------------------------------------
%	COMMUNICATION SKILLS SECTION
%----------------------------------------------------------------------------------------


%----------------------------------------------------------------------------------------
%	LANGUAGES SECTION
%----------------------------------------------------------------------------------------

%\section{Languages}

%\cvitemwithcomment{English}{Fluent}{}
%\cvitemwithcomment{Hindi}{Fluent}{}
%\cvitemwithcomment{French}{Intermediate}{}
%\cvitemwithcomment{Dutch}{Basic}{Basic words and phrases only}


%----------------------------------------------------------------------------------------
%	COVER LETTER
%----------------------------------------------------------------------------------------

% To remove the cover letter, comment out this entire block

%\clearpage

%\recipient{HR Department}{Corporation\\123 Pleasant Lane\\12345 City, State} % Letter recipient
%\date{\today} % Letter date
%\opening{Dear Sir or Madam,} % Opening greeting
%\closing{Sincerely yours,} % Closing phrase
%\enclosure[Attached]{curriculum vit\ae{}} % List of enclosed documents

%\makelettertitle % Print letter title

%\lipsum[1-3] % Dummy text

%\makeletterclosing % Print letter signature

%----------------------------------------------------------------------------------------

%\nocite{*}
%\bibliographystyle{apsrev4-1}
%\bibliography{pub}
\end{document}



